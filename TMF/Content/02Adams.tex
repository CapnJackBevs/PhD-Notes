\newpage
\section{The Adams Spectral Sequence}
\quote{``It has been suggested that the name ‘spectral’ was given because, like spectres, spectral sequences are terrifying, evil, and dangerous. I have heard no one disagree with this interpretation, which is perhaps not surprising since I just made it up"
\\\textit{`Spectral Sequences, Friends or foe?', Ravi Vakil}}

This section is a short introduction to the Adams spectral sequence, which is a tool that computes the homotopy groups $\pi_n(X)$ of a spectrum $R$ from its $\mathcal{E}$-(co)homology for some (co)homology theory $\mathcal{E}$ (usually $\mathcal{E}=H\mathbb{F}_2$, i.e. ordinary (co)homology with $\mathbb{F}_2=\mathbb{Z}/2\mathbb{Z}$ coefficients). The first step is to give an Adam's filtration of our spectrum $R$, i.e. a sequence 
	\[
	R_n \to R_{n-1} \to \dots \to R_1 \to R_0 := R
	\]
such that $R_n$ maps onto a space $K_n$ given by the wedge of suspensions of the spectrum $\mathcal{E}$, with this map $R_n \to K_n$ inducing an injection on $\mathcal{E}$-homology (resp. a surjection on $mathcal{E}$-cohomolgy) and with $X_{n+1} = \text{fib}(X_n \to K_n)$. We will start by giving several differrent interpretations of the Adams Spectral sequence, each have their own benefits and the reader may decide which one they prefer, it shouldn't matter as the author will avoid any spectral sequence calcualtions anyway, but it's nice to know that we could.


\newpage
\subsection{Idea: ASS as Descent}
%Suppose we have a pair of adjoint functors $F : \mathcal{C} \leftrightarrow \mathcal{D} : U$ with unit $\eta : 1_C \Rightarrow UF$, 
We first recall what descent is in terms of descent for monads of a category $\mathcal{C}$ (see for instance \cite{Hess18} or \cite{Mesablishvili06}). Given a monad\footnote{i.e. and endofunctor $T : \mathcal{C} \to \mathcal{C}$ along with `composition' $\mu : T^2 \Rightarrow T$ and `unit' $\eta : 1_C \to T$ natural transformations.} $(T, \mu, \eta)$ recall the Eilenberg-Moore category $\text{Alg}_\mathcal{C}(T)$ (or `category of $T$-algebras') is the category of pairs $(c,g)$ with $c \in \text{ob}\mathcal{C}$ and a associative and unital $g \in \text{Hom}_\mathcal{C}(Tc,c)$ (i.e. $g \circ Tg = g \circ \mu_c$ and $g \circ \eta_c = 1_c$) and morphsisms $(c,g) \to (c',g')$ given by linear $f : c \to c'$, i.e. $g' \circ Tg = g \circ m$. The forgetful functor $U : \text{Alg}_\mathcal{C}(T) \to \mathcal{C}$ admits a right adjoint $F : C \to \text{Alg}_\mathcal{C}(T)$, allowing us to talk about the `free algebra' $(Fc, \mu_c)$ generated by an object $c \in \text{ob}(\mathcal{C})$. 

The Eilenberg-Moore category $\text{Alg}_\mathcal{C}(T)$ allows us to realise any monad as coming from the associated forgetful-free adjunction just described, thus we can think of monads and adjunctions as being one in the same, but we can now come up with `comparison functors' between a category $\mathcal{D}$ and the monad associated to an adjunction $F : \mathcal{C} \leftrightarrows \mathcal{D} : U$ in hope that $\mathcal{D}$ might be a nicer category to work with than the category $\text{Alg}_\mathcal{C}^T$ of $T = UF$-algebras, this is the content of Beck's monadicity theorems. We can get a dual theory of \textit{comonads} and \textit{coalgebras} by reading the last two paragraphs backwards, or something like that.

\begin{definition}{def:descent-category}
The \textit{descent category} $\mathbb{D}(T)$ of a monad $T : \mathcal{C} \to \mathcal{C}$ is the category $\text{CoAlg}(\text{Alg}(T))$ of $K^T$-coalgeras of $T$-algebras.
\end{definition}

Unpacking that a bit we have that we can associate an adjunction $F^T : C \leftrightarrows \text{Alg}_\mathcal{C}(T) : U^T$ to $T$ and consider the comonad $K := F^TU^T$, objects of $\mathbb{D}(T)$ are coalgebras of the comonad $K$. Objects of $\mathbb{D}(T)$ are called \textit{descent data} and $T$ is said to \textit{satisfy descent} if $\mathcal{C}$ sits inside $\mathbb{D}(T)$, i.e. if the functor $\mathcal{C} \to \mathbb{D}(T)$ given by $c \mapsto (F^Tc, F^T\eta_c)$ is fully-faithful. Thus descent datum for $(T, \mu, \eta)$ consists of a triple $(c,f,g)$ with $c \in \text{ob}\mathcal{C}$, associative and unital $f : Tc \to c$ and coassociative an counital $g : c \to Tc$ fitting into a diagram
	\begin{figure}[ht!]
	\centering
	\begin{tikzcd}
		Tc \arrow[r,"f"] \arrow[d,"Tg"']	& c \arrow[d,"g"] \\
		T^2c \arrow[r,"\mu_c"']			& Tc \\
	\end{tikzcd}
	\end{figure}
\vspace{-5ex} 

Note that every object $c \in \text{ob}\mathcal{C}$ has \textit{canonical descent data} given by $(Tc,\mu_c,T\eta_c)$.

\begin{env}[Sheaves and Descent]{Example}{green!20}{ex:sheaves-descent}
Let $X$ be a topological space and $\mathcal{U}$ be a cover of $X$; $X = \bigcup_{U \in \mathcal{U}} U$. We let $\mathcal{C} = \text{PSh}(X) := \text{Fun}(\text{open}(X)^\text{op},\text{Set})$ be the category of presheaves of sets on $X$. We get a monad $T^\mathcal{U} : \text{PSh}(X) \to \text{PSh}(X)$ from the cover $\mathcal{U}$ by sending a presheaf $\mathcal{F}$ to $T^\mathcal{U}(\mathcal{F}) : V \mapsto \underset{\mathcal{U} \ni U \subseteq V}{\text{colim}} \mathcal{F}(U)$. The category of algebras $\text{Alg}_{\text{PSh}(X)}(T^\mathcal{U})$ is the category of sheaves since sheafifying a sheaf has no effect.

The descent datum of the `$\mathcal{U}$-sheafification monad' $T^\mathcal{U}$ is given by a triple $(\mathcal{F}, f, g)$ with $\mathcal{F}$ a presheaf, $f : T^\mathcal{U}(\mathcal{F}) \to \mathcal{F}$ given by maps
	\[
	\left\{
	T^\mathcal{U}(\mathcal{F})(U)
	\overset{f|_U}{\to}
	\mathcal{F}(U)
	: U \in \mathcal{U}
	\right\}
	=
	\left\{
	(X_U \in \mathcal{F}(U))_{U\in\mathcal{U}}
	: X_U = X_V \text{ on } U \cap V
	\right\},
	\]
and a map $g : \mathcal{F} \to T^\mathcal{U}(\mathcal{F})$ which, along with the commutative square above, translates into the sheaf condition (exercise; n.B. $Tg = \mu_\mathcal{F}$ are the identity since sheafifying a sheaf gives a sheaf). We think of $f$ as `breaking' $\mathcal{F}$ up into parts determined by $\mathcal{U}$, and $g$ as gluing these pieces back together again.
\end{env}

We may extend this to the world of $\infty$-categories by taking Adam's (co)bar construction associating a simplicial object $T^\bullet$ in $\mathcal{C}$ to the monad $T$, but we leave that discussion to \cite{Hess18}, and give instead a `proof by example' for the Adams spectral sequence.



\newpage
Let $\mathcal{E}$ be an $\mathbb{E}_\infty$-ring so that the unique map $\mathbb{S} \to \mathcal{E}$ makes $\mathcal{E}$ into an $\mathbb{S}$-module where $\mathbb{S}$ is the sphere spectrum. The category $\text{Sp}_{\geq0}$ of $\mathbb{E}_\infty$-rings has a monoid structure given by the smash product of spectra $-\otimes_\mathbb{S} - := -\wedge-$, giving us a map $-\otimes_\mathbb{S} \mathcal{E} : \text{Mod}_\mathbb{S} \to \text{Mod}_\mathcal{E}$ with right adjoint given by the forgetful functor. The adjunction $-\otimes_\mathbb{S} \mathcal{E} : \text{Mod}_\mathbb{S} \leftrightarrows \text{Mod}_\mathcal{E} : U$ leads to a monad $U(-\otimes_\mathbb{S} \mathcal{E})$, so we can use the above paradigm to find descent data.

This is a good idea since typically it is hard to compute the homotopy groups $\pi_\ast(R)$ of an $\mathbb{E}_\infty$-ring $R$, so by tensoring with $\mathcal{E}$ and taking homotopy groups we are reduced to finding the $\mathcal{E}$-homology $\mathcal{E}_\ast R:= \pi_\ast(\mathcal{E}\otimes R)$ of $R$. The question remains; can we recover the homotopy groups $\pi_\ast(R)$ from the $\mathcal{E}$-homology of $R$? The answer is given by the Adams spectral sequence - in short it says ``yes, up to $\mathcal{E}$-completion".

To do this we create a cosimplicial resolution via the cobar $\text{Bar}_n(R) := R^{\otimes n+1} \otimes \mathcal{E}$, and our hope is that the (homotopy) limit $\text{Tot}^\mathcal{E}_\bullet(R) := \lim_{n\geq0}\text{Bar}_n(R)$ is equivalent to our $\mathbb{E}_\infty$-ring $R$. If $\mathcal{E} = H\mathbb{F}_p$ is ordinary $\mathbb{F}_p$-homology then this is the \textit{classical Adams Spectral sequence}, and if $R = MU$ is complex cobordism then this recovers the \textit{Adams-Novikov Spectral sequence}. 


{\color{orange}
	\textbf{Note:} This section comes from \href{https://amathew.wordpress.com/2012/09/21/3844/}{this blog post}. I will suppliment it later with Mike Hopkin's notes on `Complex Oriented Cohomology Theories and the Language of Stacks' (Course 18.917, notes \href{https://ncatlab.org/nlab/files/HopkinsLecture.pdf}{here}) and on Hess' paper `A general framework for homotopic descent and codescent' (\href{https://arxiv.org/abs/1001.1556}{arXiv:1001.1556})
}

% Thom spectrum of BO(n) gives the oriented cobordism spectrum, we can calculate the homotopy groups of this spectrum entirely using its Z/2Z homology, which we can calculate via its steenrod algebra structure.


\newpage
\subsection{Idea: ASS as Exact Couples}
This section is based on \href{https://ncatlab.org/nlab/show/spectral+sequences+in+homotopy+type+theory}{this blog post}. Classically the Adam's spectral sequence comes from `derived couples', and we'll see a bit of that in this section but it's easy for the bulk of the work to be obfiscated in the details and to lose track of what's going on. In this section we'll see spectral sequences for what they are; a mild generalisation of exact sequences. 

The notion of spectral sequences as descent from the previous section is a useful way of understanding what spectral sequences \textit{do}, but it's not so clear (at least for this author) how one takes that perspective and applies it to make computations. The idea essentially comes down to the interplay between filtrations $\cdots F_2R \subseteq F_1R \subseteq F_0R =: R$ of a ring $R$ and the associated graded ring $\bigoplus_{n=0}^\infty F_nR$, after all, filtrations are something we can get a hold of in algebraic topology. 

Of course, we have a filtration already; given our spectrum $R$ and a cohomology theory $\mathcal{E}$ we have a filtration $\text{Tot}_\bullet^\mathcal{E}(R) := \lim_{n\geq0} \left( R^{\otimes n} \otimes \mathcal{E} \right)$. 



\newpage
\subsection{Idea: ASS from the heart}
Lurie's higher algebra \cite{HA} gives us a new way to view spectral sequenecs again. 


\subsection{Idea: ASS as a graded ring}
See (not sure actually, Wiebel? Eisenbud?)




\subsection{Bringing it all together}
This has been fun, but now we look at how to realte all these different ways of understanding the Adams spectral sequence before moving on to some example usages of it and, eventually, using it to calculate the homotopy groups of the tmf spectral sequence. 



